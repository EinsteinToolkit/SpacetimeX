% *======================================================================*
%  Cactus Thorn template for ThornGuide documentation
%  Author: Ian Kelley
%  Date: Sun Jun 02, 2002
%  $Header$
%
%  Thorn documentation in the latex file doc/documentation.tex
%  will be included in ThornGuides built with the Cactus make system.
%  The scripts employed by the make system automatically include
%  pages about variables, parameters and scheduling parsed from the
%  relevant thorn CCL files.
%
%  This template contains guidelines which help to assure that your
%  documentation will be correctly added to ThornGuides. More
%  information is available in the Cactus UsersGuide.
%
%  Guidelines:
%   - Do not change anything before the line
%       % START CACTUS THORNGUIDE",
%     except for filling in the title, author, date, etc. fields.
%        - Each of these fields should only be on ONE line.
%        - Author names should be separated with a \\ or a comma.
%   - You can define your own macros, but they must appear after
%     the START CACTUS THORNGUIDE line, and must not redefine standard
%     latex commands.
%   - To avoid name clashes with other thorns, 'labels', 'citations',
%     'references', and 'image' names should conform to the following
%     convention:
%       ARRANGEMENT_THORN_LABEL
%     For example, an image wave.eps in the arrangement CactusWave and
%     thorn WaveToyC should be renamed to CactusWave_WaveToyC_wave.eps
%   - Graphics should only be included using the graphicx package.
%     More specifically, with the "\includegraphics" command.  Do
%     not specify any graphic file extensions in your .tex file. This
%     will allow us to create a PDF version of the ThornGuide
%     via pdflatex.
%   - References should be included with the latex "\bibitem" command.
%   - Use \begin{abstract}...\end{abstract} instead of \abstract{...}
%   - Do not use \appendix, instead include any appendices you need as
%     standard sections.
%   - For the benefit of our Perl scripts, and for future extensions,
%     please use simple latex.
%
% *======================================================================*
%
% Example of including a graphic image:
%    \begin{figure}[ht]
% 	\begin{center}
%    	   \includegraphics[width=6cm]{MyArrangement_MyThorn_MyFigure}
% 	\end{center}
% 	\caption{Illustration of this and that}
% 	\label{MyArrangement_MyThorn_MyLabel}
%    \end{figure}
%
% Example of using a label:
%   \label{MyArrangement_MyThorn_MyLabel}
%
% Example of a citation:
%    \cite{MyArrangement_MyThorn_Author99}
%
% Example of including a reference
%   \bibitem{MyArrangement_MyThorn_Author99}
%   {J. Author, {\em The Title of the Book, Journal, or periodical}, 1 (1999),
%   1--16. {\tt http://www.nowhere.com/}}
%
% *======================================================================*

% If you are using CVS use this line to give version information
% $Header$

\documentclass{article}

% Use the Cactus ThornGuide style file
% (Automatically used from Cactus distribution, if you have a
%  thorn without the Cactus Flesh download this from the Cactus
%  homepage at www.cactuscode.org)
\usepackage{../../../../doc/latex/cactus}

\begin{document}

% The author of the documentation
\author{Cheng-Hsin Cheng,
  Jake Doherty,
  Michail Chabanov,
	Alexandru Dima
	}

% The title of the document (not necessarily the name of the Thorn)
\title{NewRadX}

% the date your document was last changed, if your document is in CVS,
% please use:
%    \date{$ $Date$ $}
\date{April 17 2024}

\maketitle

% Do not delete next line
% START CACTUS THORNGUIDE

% Add all definitions used in this documentation here
%   \def\mydef etc

% Add an abstract for this thorn's documentation
\begin{abstract}
This thorn implements radiative boundary conditions for \code{CarpetX}. The
original implementation was introduced in the \code{Carpet}-based \code{NewRad}
thorn in the \code{EinsteinEvolve}~\cite{SpacetimeX_NewRadX_EinsteinEvolve}
arrangement.
\end{abstract}

% The following sections are suggestive only.
% Remove them or add your own.

\section{Introduction}
This thorn provides routines to implement radiative boundary conditions as
described in section VI of~\cite{SpacetimeX_NewRadX_Alcubierre:2002kk}:
\begin{equation}
f = f_0 + u(r - v t)/r\mbox{,}
\label{eqn:SpacetimeX_NewRadX_bc}
\end{equation}
in its differential form
\begin{equation}
\partial_t f + v \partial_r f - v\,(f - f_0) = 0\mbox{,}
\label{eqn:SpacetimeX_NewRadX_diff_bc}
\end{equation}
where $v$ is the asymptotic propagation speed of the field $f$ and $f_0$
is its asymptotic value.

\section{Physical System}
% RH: this is copied verbatim from Alcubierre:2002kk
Quoting~\cite{SpacetimeX_NewRadX_Alcubierre:2002kk}:

At the outer boundary we use a radiation (Sommerfeld) boundary
condition.  We start from the assumption that near the boundary all
fields behave as spherical waves, namely we impose the condition
\begin{equation}
f = f_0 + u(r-vt)/r .
\label{eq:radiation}
\end{equation}
Where $f_0$ is the asymptotic value of a given dynamical variable
(typically 1 for the lapse and diagonal metric components, and zero
for everything else), and $v$ is some wave speed.  If our
boundary is sufficiently far away one can safely assume that the speed
of light is 1, so $v=1$ for most fields.  However, the gauge variables
can easily propagate with a different speed implying a different value
of $v$.

In practice, we do not use the boundary condition~(\ref{eq:radiation})
as it stands, but rather we use it in differential form:
\begin{equation}
\partial_t f + v \partial_r f - v \, (f - f_0)/r = 0 .
\end{equation}
Since our code is written in Cartesian coordinates, we transform the
last condition to
\begin{equation}
\frac{x_i}{r} \, \partial_t f + v \partial_i f + \frac{v x_i}{r^2} \,
\left( f - f_0 \right) = 0 .
\end{equation}
We finite difference this condition consistently to second order in
both space and time and apply it to all dynamic variables (with
possible different values of $f_0$ and $v$) at all boundaries.

There is a final subtlety in our boundary treatment.  Wave propagation
is not the only reason why fields evolve near a boundary.  Simple
infall of the coordinate observers will cause some small evolution as
well, and such evolution is poorly modeled by a propagating wave. This
is particularly important at early times, when the radiative boundary
condition introduces a bad transient effect. In order to minimize the
error at our boundaries introduced by such non-wavelike evolution, we
allow for boundary behavior of the form:
\begin{equation}
f = f_0 + u(r-vt)/r + h(t)/r^n ,
\label{eq:radpower}
\end{equation}
with $h$ a function of $t$ alone and $n$ some unknown power. This
leads to the differential equation
\begin{eqnarray}
\partial_t f + v \partial_r f
- \frac{v}{r} \, (f - f_0) &=&
\frac{v h(t)}{r^{n+1}} \left( 1 - n v \right)
+ \frac{h'(t)}{r^n} \nonumber \\
&\simeq& \frac{h'(t)}{r^n} \qquad \mathrm{for\ large\ } r ,
\end{eqnarray}
or in Cartesian coordinates
\begin{equation}
\frac{x_i}{r} \, \partial_t f + v \partial_i f + \frac{v x_i}{r^2} \,
\left( f - f_0 \right) \simeq \frac{x_i h'(t)}{r^{n+1}} .
\label{eqn:SpacetimeX_NewRadX_num_bc}
\end{equation}

This expression still contains the unknown function $h'(t)$. Having
chosen a value of $n$, one can evaluate the above expression one point
away from the boundary to solve for $h'(t)$, and then use this value
at the boundary itself.  Empirically, we have found that taking $n=3$
almost completely eliminates the bad transient caused by the radiative
boundary condition on its own.

\section{Numerical Implementation}
The thorn implements the boundary condition
~\eqref{eqn:SpacetimeX_NewRadX_num_bc} for the evolution.  The original
\code{Carpet}-based version of \code{NewRad} provides an additional
routine \code{ExtrapolateGammas}, which can be used to extrapolate
a grid function from the interior to the outer boundaries at $t=0$.
Originally, this was used in the BSSN evolution thorn \code{ML\_BSSN}
on the contracted Christoffel symbols
$g^{jk} \Gamma^i_{jk}$ (\code{Xt1}\ldots\code{Xt3} in the code).
This extrapolation routine is not yet ported over to the current version
of \code{NewRadX}.


% \subsection{Extrapolation}
% In the \code{CarpetX} version of NewRadX, the boundary information is
% obtained from the \code{Loop} thorn.
%
% At outer boundary points, as determined by \code{GenericFD}'s routine
% \code{GenericFD\_GetBoundaryInfo} we use cubic polynomials along the
% normal of the boundary to extrapolate data from the interior into the
% boundary region. Due to the use of cubic polynomial we require at least
% four (4) points of valid data to be available for the extrapolation.
% This \emph{includes} ghost zones, which are assumed to contain valid
% data in the interior of the domain.

\subsection{Radiative boundary condition}
The derivatives $\partial_i f$ appearing in
\eqref{eqn:SpacetimeX_NewRadX_num_bc} are approximated as
2\textsuperscript{nd} order accurate finite difference expressions using
centred expressions
\begin{equation}
f'(x) \approx \frac{f(x-\Delta x) - f(x+\Delta x)}{2\Delta x}
\end{equation}
and one sided expressions
\begin{equation}
f'(x) \approx \pm\frac{3 f(x) - 4 f(x\mp\Delta x) + f(x\mp2\Delta
x)}{2\Delta x}\mbox{,}
\end{equation}
for boundaries around the $\pm$ coordinate direction.

% copied from newrad.cc so likely originally written by Miguel
% Alcubierre.
Finally we try to extrapolate for the part of the boundary
that does not behave as a pure wave (i.e.\ Coulomb type
terms caused by infall of the coordinate lines).

This we do by comparing the source term \code{nghostzones} grid points
away from the boundary (which we already have), to what
we would have obtained if we had used the boundary
condition there. The difference gives us an idea of the
missing part and we extrapolate that to the boundary
assuming a power-law decay.

\section{Using This Thorn}
The thorn provides the routine \code{NewRadX\_Apply}, which should be scheduled
in the \code{ODESolvers\_RHS} bin to apply radiative boundary conditions on
evolved variables.  To use it in an evolution thorn, it can either be called
inside the RHS evaluation routine, or scheduled in a separate function right
after the RHS routine.

The call signature is:
\begin{verbatim}
void NewRadX_Apply                               \
        (const cGH *restrict const cctkGH,       \
         const Loop::GF3D2<const CCTK_REAL> var, \
         Loop::GF3D2<CCTK_REAL> rhs,             \
         CCTK_REAL var0,                         \
         CCTK_REAL v0,                           \
         CCTK_REAL radpower)
\end{verbatim}
where \code{var0} corresponds to $f_0$, \code{v0} is $v$ and
\code{radpower} is the assumed decay exponent $n$.

\subsection{Obtaining This Thorn}
The thorn is part of the \code{SpacetimeX} arrangement available on
GitHub.
\begin{verbatim}
git clone git@github.com:lwJi/SpacetimeX.git
\end{verbatim}

\subsection{Basic Usage}
For an overview of the implementation and usage of this thorn, please
see~\cite{SpacetimeX_NewRadX_intro_presentation} for a presentation by
Alexandru Dima and Cheng-Hsin Cheng during the weekly \code{CarpetX}
developer meetings.


\subsection{Special Behaviour}
In contrast with the \code{Carpet}-version of \code{NewRad}, which updates
the RHS variables on the physical outer boundary of the simulation domain,
\code{NewRadX} updates them on the so-called ``outermost interior points''.
These points are defined as the interior points which are located
\code{nghostzones} or fewer grid points from the outer boundary.
Therefore, the boundary conditions are effectively applied on a set of points
that are labelled as ``interior'' in \code{CarpetX}.
It is necessary to introduce this change because imposing a wave-like time
evolution on the \code{CarpetX} ``proper'' boundary points is incompatible with the
synchronization conducted by AMReX.


Currently, this thorn only supports single-patch, Cartesian grids provided by
\code{CarpetX}, but not multi-patch coordinate systems.


\subsection{Interaction With Other Thorns}
This thorn requires thorn \code{Loop} which is part of
\code{CarpetX}.

\subsection{Examples}
The simplest example is likely to inspect the source code for
\code{TestNewRadX}, which is a thorn from \code{SpacetimeX} that tests the
boundary condition on the evolution of a scalar pulse on flat background.
Here the boundary conditions are applied in the RHS calculation routine
\code{TestNewRadX\_RHS}, where \code{NewRadX\_Apply} is called after the main
loop that updates the RHS of the state variables in the interior.

\begin{verbatim}
<configuration.ccl>
REQUIRES Loop NewRadX
\end{verbatim}

\begin{verbatim}
<interface.ccl>
USES INCLUDE HEADER: newradx.hxx
\end{verbatim}

\begin{verbatim}
<schedule.ccl>
SCHEDULE TestNewRadX_RHS IN ODESolvers_RHS
{
  LANG: C
  READS: state(everywhere)
  WRITES: rhs(interior)
  SYNC: rhs
} "Calculate scalar wave RHS"
\end{verbatim}

\begin{verbatim}
<test_newradx.cxx>
#include <newradx.hxx>

extern "C" void TestNewRadX_RHS(CCTK_ARGUMENTS) {
  DECLARE_CCTK_ARGUMENTSX_TestNewRadX_RHS;
  DECLARE_CCTK_PARAMETERS;

  /* First loop over the interior to update RHS of the state variables */
  grid.loop_int_device<0, 0, 0>(grid.nghostzones,
                                [=] CCTK_DEVICE(const Loop::PointDesc &p)
                                    CCTK_ATTRIBUTE_ALWAYS_INLINE {
                                      uu_rhs(p.I) = vv(p.I);
                                      vv_rhs(p.I) = Laplacian_4thorder(uu, p);
                                    });

  /* After RHS grid functions are updated, apply boundary conditions */
  if (test_use_newradx) {
    NewRadX::NewRadX_Apply(cctkGH, uu, uu_rhs, 0, 1, n_falloff);
    NewRadX::NewRadX_Apply(cctkGH, vv, vv_rhs, 0, 1, n_falloff + 1);
  }

}
\end{verbatim}


Other examples of \code{CarpetX} evolution thorns that use \code{NewRadX}
include \code{CanudaX\_BSSNMoL}\cite{SpacetimeX_NewRadX_CanudaX_BSSNMoL}
and \code{Z4c}\cite{SpacetimeX_NewRadX_Z4c}.


\subsection{Support and Feedback}
Please use the Einstein Toolkit mailing list
\url{users@einsteintoolkit.org} to report issues and ask for help.

\section{History}

\subsection{Thorn Source Code}
\begin{itemize}
\item Initial version of method likely by Miguel Alcubierre;
\item \code{Carpet} version of code by Erik Schnetter;
\item \code{CarpetX} version of code by Cheng-Hsin Cheng, Jake Doherty,
  Michail Chabanov, Alexandru Dima.
\end{itemize}

\subsection{Thorn Documentation}
\begin{itemize}
\item Initial version by Roland Haas using text by Miguel, Erik.
\item Current version by Alexandru Dima and Cheng-Hsin Cheng.
\end{itemize}

\subsection{Acknowledgements}
This documentation is based on the original one for the \code{Carpet} version
of \code{NewRad}, which copies text from comments in the source code as well
as the paper~\cite{SpacetimeX_NewRadX_Alcubierre:2002kk}.

\begin{thebibliography}{9}

\bibitem{SpacetimeX_NewRadX_EinsteinEvolve} ``EinsteinEvolve``,
  \url{https://bitbucket.org/einsteintoolkit/einsteinevolve}.

\bibitem{SpacetimeX_NewRadX_Alcubierre:2002kk}
  M.~Alcubierre, B.~Bruegmann, P.~Diener, M.~Koppitz, D.~Pollney, E.~Seidel and R.~Takahashi,
  ``Gauge conditions for long term numerical black hole evolutions without excision,''
  Phys.\ Rev.\ D {\bf 67}, 084023 (2003)
  doi:10.1103/PhysRevD.67.084023
  [gr-qc/0206072].

\bibitem{SpacetimeX_NewRadX_intro_presentation} ``NewRadX Intro``,
  \url{https://github.com/eschnett/CarpetX/wiki/NewRadX-Intro}.

\bibitem{SpacetimeX_NewRadX_CanudaX_BSSNMoL} ``CanudaX\_Lean``,
  \url{https://bitbucket.org/canuda/canudax_lean}.

\bibitem{SpacetimeX_NewRadX_Z4c} ``Z4c'',
  \url{https://github.com/lwJi/SpacetimeX/tree/main/Z4c}.

\end{thebibliography}

% Do not delete next line
% END CACTUS THORNGUIDE

\end{document}
