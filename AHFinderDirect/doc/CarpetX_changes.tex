% *======================================================================*
%  Cactus Thorn template for ThornGuide documentation
%  Author: Ian Kelley
%  Date: Sun Jun 02, 2002
%  $Header$
%
%  Thorn documentation in the latex file doc/documentation.tex
%  will be included in ThornGuides built with the Cactus make system.
%  The scripts employed by the make system automatically include
%  pages about variables, parameters and scheduling parsed from the
%  relevent thorn CCL files.
%
%  This template contains guidelines which help to assure that your
%  documentation will be correctly added to ThornGuides. More
%  information is available in the Cactus UsersGuide.
%
%  Guidelines:
%   - Do not change anything before the line
%       % BEGIN CACTUS THORNGUIDE",
%     except for filling in the title, author, date etc. fields.
%   - You can define your own macros are OK, but they must appear after
%     the BEGIN CACTUS THORNGUIDE line, and do not redefine standard
%     latex commands.
%   - To avoid name clashes with other thorns, 'labels', 'citations',
%     'references', and 'image' names should conform to the following
%     convention:
%       ARRANGEMENT_THORN_LABEL
%     For example, an image wave.eps in the arrangement CactusWave and
%     thorn WaveToyC should be renamed to CactusWave_WaveToyC_wave.eps
%   - Graphics should only be included using the graphix package.
%     More specifically, with the "includegraphics" command. Do
%     not specify any graphic file extensions in your .tex file. This
%     will allow us (later) to create a PDF version of the ThornGuide
%     via pdflatex. |
%   - References should be included with the latex "bibitem" command.
%   - For the benefit of our Perl scripts, and for future extensions,
%     please use simple latex.
%
% *======================================================================*
%
% Example of including a graphic image:
%    \begin{figure}[ht]
% 	\begin{center}
%    	   \includegraphics[width=6cm]{MyArrangement_MyThorn_MyFigure}
% 	\end{center}
% 	\caption{Illustration of this and that}
% 	\label{MyArrangement_MyThorn_MyLabel}
%    \end{figure}
%
% Example of using a label:
%   \label{MyArrangement_MyThorn_MyLabel}
%
% Example of a citation:
%    \cite{MyArrangement_MyThorn_Author99}
%
% Example of including a reference
%   \bibitem{MyArrangement_MyThorn_Author99}
%   {J. Author, {\em The Title of the Book, Journal, or periodical}, 1 (1999),
%   1--16. {\tt http://www.nowhere.com/}}
%
% *======================================================================*

% If you are using CVS use this line to give version information
% $Header$

\documentclass{article}
\bibliographystyle{alpha}

% Use the Cactus ThornGuide style file
% (Automatically used from Cactus distribution, if you have a
%  thorn without the Cactus Flesh download this from the Cactus
%  homepage at www.cactuscode.org)
\usepackage{cactus}

\begin{document}

%%%%%%%%%%%%%%%%%%%%%%%%%%%%%%%%%%%%%%%%%%%%%%%%%%%%%%%%%%%%%%%%%%%%%%%%%%%%%%%%

% The author of the documentation
\author{Allen Wen \quad {\tt <acw6923@g.rit.edu>}}

% The title of the document (not necessarily the name of the Thorn)
\title{{\bf AHFinderDirect} -- Moving Jonathan Thornburg's Original Code to CarpetX}

% the date your document was last changed, if your document is in CVS,
% please us:
%    \date{$ $Date$ $}
\date{July 27, 2023}

\maketitle

% Do not delete next line
% START CACTUS THORNGUIDE

% Add all definitions used in this documentation here
%   \def\mydef etc

%%%%%%%%%%%%%%%%%%%%%%%%%%%%%%%%%%%%%%%%

% misc text stuff
\def\code#1{{\tt #1}}			% variable names, parameter values, etc
\def\defn#1{``#1''}			% definition of a term
\def\arrangement#1{{\bf #1}}		% name of an arrangement
\def\thorn#1{{\bf #1}}			% name of a thorn
\def\program#1{{\bf #1}}		% name of computer program
\def\cvsplace#1{{\bf #1}}		% name of a CVS repository/directory/tag
\def\cf{\hbox{cf.\hbox{}}}
\def\eg{\hbox{eg.\hbox{}}}
\def\ie{\hbox{i.e.\hbox{}}}
\def\eqref#1{$(\ref{#1})$}

% get size/spacing of "++" right, cf online C++ FAQ question 35.1
\def\Cplusplus{\hbox{C\raise.25ex\hbox{\footnotesize ++}}}

% misc math mode stuff
\def\assign{\leftarrow}			% for pseudocode algorithms
\def\ltsim{\lesssim}
\def\gtsim{\gtrsim}
\def\tfrac#1#2{{\textstyle    \frac{#1}{#2}}}
\def\dfrac#1#2{{\displaystyle \frac{#1}{#2}}}
\def\thalf{\tfrac{1}{2}}
\def\const{{\rm const}}
\def\ij{{ij}}
\def\uv{{uv}}
\def\del{\nabla}

\def\I{{\text{\scriptsize I}}}                  % grid-point index
\def\J{{\text{\scriptsize J}}}                  % grid-point index
\def\K{{\text{\scriptsize K}}}                  % grid-point index
\def\M{{\text{\scriptsize M}}}                  % molecule index

\def\Jac[#1]{{\bf J} \bigl[ #1 \bigr]}          % discrete Jacobian for fn #1

\section{Features Removed}
\begin{enumerate}
\item \textbf{Removed Masking:}
    All references to the \texttt{SpaceMask} thorn have been removed and the \texttt{ahmask} gridfunction has been manually set to 0 everywhere. \textit{Must fix if this feature is desired.}

\item \textbf{\texttt{StaticConformal} Support:}
    All references to the \texttt{StaticConformal} thorn have been  removed.

\item \textbf{Symmetries Provided by \texttt{GRID}:}
    The new \texttt{Coordinates} thorn currently does not set \texttt{domain, bitant\_plane, quadrant\_direction,} or \texttt{rotation\_axis} so these parameters are no longer used to set up the \texttt{AHFinder} patch system. \textit{If these features are ever added, this will need to be fixed.}
    
\end{enumerate}

\section{Changes Made to Original Code}
\begin{enumerate}
\item \textbf{Use CarpetX Interpolator:}
    \texttt{CCTK\_InterpGridArrays} passes arguments to \texttt{ CarpetX\_DriverInterpolate} and then \texttt{ CarpetX\_Interpolate}, which does not use an interpolator handle or a coordinate system handle. Interpolator handles are still provided by \texttt{AEILocalInterp} but the coordinate system handle has been manually set to 0.

\item \textbf{Scheduling Differences Between \texttt{Carpet} and \texttt{CarpetX}:}
    \texttt{CCTK\_BASEGRID} now runs after each regrid. This was not expected by \texttt{AHFinderDirect} so a flag was added to ensure the setup function only runs once.

\item \textbf{Cactus Reductions Replaced by MPI Calls}
    \texttt{CarpetX} does not have the reduction handles and wrapper functions the \texttt{Carpet} had. Calls to \texttt{MPI\_AllReduce} were used instead.

\end{enumerate}

\section{Additions to \texttt{SpacetimeX}}
The thorns \texttt{AEILocalInterp} and \texttt{SphericalSurface} have been added. They worked with \texttt{AEILocalInterp} quite easily but additional optimization or adaptation might be needed to work with other thorns.

\section{Example}
Here is a CarpetX thornlist that runs a binary black hole simulation:

\begin{verbatim}

# run.me:
# run.cores: 40
# run.memory: 1.0e9
# run.time: 7200.0

ActiveThorns = "
ADMBaseX
AEILocalInterp
AHFinderDirect
CarpetX
Coordinates
BoxInBox
Formaline
IOUtil
ODESolvers
SystemTopology
SphericalSurface
TimerReport
TmunuBaseX
TwoPunctures
Weyl
Z4c
PunctureTracker
#TODO Z4cNRPy
"

# domain_size = 128.0
# domain_spacing = 1.0
# fine_box_size = 2.0
# fine_box_spacing = 0.0416666666667
# dtfac = 0.25

Cactus::cctk_show_schedule = yes

Cactus::presync_mode = "mixed-error"

Cactus::terminate = "runtime"
Cactus::cctk_itlast = 1000000000
Cactus::cctk_final_time = 300.0
# Cactus::max_runtime = 30   # minutes
Cactus::max_runtime = 5 * 60 + 30   # minutes

CarpetX::verbose = yes
CarpetX::poison_undefined_values = no

CarpetX::xmin = -400
CarpetX::ymin = -400
CarpetX::zmin = -400

CarpetX::xmax = 400
CarpetX::ymax = 400
CarpetX::zmax = 400

CarpetX::ncells_x = 200
CarpetX::ncells_y = 200
CarpetX::ncells_z = 200

CarpetX::max_tile_size_x = 1024000
CarpetX::max_tile_size_y = 4
CarpetX::max_tile_size_z = 4

CarpetX::boundary_x = "dirichlet"
CarpetX::boundary_y = "dirichlet"
CarpetX::boundary_z = "dirichlet"
CarpetX::boundary_upper_x = "dirichlet"
CarpetX::boundary_upper_y = "dirichlet"
CarpetX::boundary_upper_z = "dirichlet"

CarpetX::ghost_size = 3
CarpetX::interpolation_order = 3

# -------------------- BoxInBox -------------------------------------------------
CarpetX::max_num_levels = 11
CarpetX::regrid_every = 32
CarpetX::regrid_error_threshold = 0.9

PunctureTracker::track_boxes = yes

BoxInBox::num_regions = 1

# Region 1
BoxInBox::shape_1 = "cube"
BoxInBox::num_levels_1 = 9
BoxInBox::position_x_1 = +4
BoxInBox::radius_1[1] = 220
BoxInBox::radius_1[2] = 110
BoxInBox::radius_1[3] = 55
BoxInBox::radius_1[4] = 25
BoxInBox::radius_1[5] = 10
BoxInBox::radius_1[6] = 5
BoxInBox::radius_1[7] = 2
BoxInBox::radius_1[8] = 1

# Region 2
BoxInBox::shape_2 = "cube"
BoxInBox::num_levels_2 = 9
BoxInBox::position_x_2 = -4
BoxInBox::radius_2[1] = 220
BoxInBox::radius_2[2] = 110
BoxInBox::radius_2[3] = 55
BoxInBox::radius_2[4] = 25
BoxInBox::radius_2[5] = 10
BoxInBox::radius_2[6] = 5
BoxInBox::radius_2[7] = 2
BoxInBox::radius_2[8] = 1

# -------------------- Multipole -------------------------------------------------
# Multipole::radius[0] = 20.0
# Multipole::variables = "WeylScal4::Psi4r{sw=-2 cmplx='Weyl::Psi4i' name='Psi4'}"
# Multipole::interpolator_pars = "order=3"
# Multipole::l_max = 4
# Multipole::verbose = "yes"
#Multipole::enable_test = "yes"

CarpetX::prolongation_type = "ddf"
CarpetX::prolongation_order = 5

ODESolvers::verbose = no
ODESolvers::method = "RK4"
CarpetX::dtfac = 0.25

ADMBaseX::initial_data = "TwoPunctures"
ADMBaseX::initial_lapse = "TwoPunctures-averaged"

# QC-0 setup
TwoPunctures::par_b = 4
TwoPunctures::par_m_plus =  0.4824
TwoPunctures::par_m_minus =  0.4824
TwoPunctures::par_P_plus [1] = +0.114
TwoPunctures::par_P_minus[1] = -0.114

TwoPunctures::grid_setup_method = "evaluation"

TwoPunctures::TP_epsilon = 1.0e-2
TwoPunctures::TP_Tiny    = 1.0e-2

TwoPunctures::verbose = yes

Z4c::calc_ADM_vars = yes                  # for Weyl
Z4c::calc_ADMRHS_vars = yes               # for Weyl
Z4c::calc_constraints = yes

Z4c::chi_floor = 1.0e-6
Z4c::alphaG_floor = 1.0e-8
Z4c::epsdiss = 0.32

# Puncture Tracker
PunctureTracker::verbose = yes

PunctureTracker::interp_order = 3

PunctureTracker::track    [0] = yes
PunctureTracker::initial_x[0] = +4

PunctureTracker::track    [1] = yes
PunctureTracker::initial_x[1] = -4

#TODO # Z4cNRPy::chi_floor = 1.0e-6
#TODO # Z4cNRPy::alphaG_floor = 1.0e-8
#TODO # Z4cNRPy::eps_diss = 0.34
#

# AHFinderDirect::print_timing_stats = "true"

# AHFinderDirect::output_Theta_every = 1
# AHFinderDirect::h_base_file_name     = "Kerr.h"
# AHFinderDirect::Theta_base_file_name = "Kerr.Theta"
#

AHFinderDirect::find_every   = 4
AHFinderDirect::move_origins = yes

AHFinderDirect::N_horizons = 2
AHFinderDirect::origin_x[1] = 4.0
AHFinderDirect::origin_y[1] = 0.0
AHFinderDirect::origin_z[1] = 0.0

AHFinderDirect::origin_x[2] = -4.0
AHFinderDirect::origin_y[2] = 0.0
AHFinderDirect::origin_z[2] = 0.0

AHFinderDirect::initial_guess_method[1] = "coordinate sphere"
AHFinderDirect::initial_guess__coord_sphere__x_center[1] =  4.0
AHFinderDirect::initial_guess__coord_sphere__y_center[1] =  0.0
AHFinderDirect::initial_guess__coord_sphere__z_center[1] =  0.0
AHFinderDirect::initial_guess__coord_sphere__radius[1] = 0.5

AHFinderDirect::initial_guess_method[2] = "coordinate sphere"
AHFinderDirect::initial_guess__coord_sphere__x_center[2] =  -4.0
AHFinderDirect::initial_guess__coord_sphere__y_center[2] =  0.0
AHFinderDirect::initial_guess__coord_sphere__z_center[2] =  0.0
AHFinderDirect::initial_guess__coord_sphere__radius[2] = 0.5

AHFinderDirect::geometry_interpolator_pars = "order=3 boundary_off_centering_tolerance={1.0e-10 1.0e-10 1.0e-10 1.0e-10 1.0e-10 1.0e-10} boundary_extrapolation_tolerance={0.0 0.0 0.0 0.0 0.0 0.0}"

IO::out_dir = $parfile
IO::out_every = 1024
IO::out_mode = "np"                       # "proc", "np", "onefile"
IO::out_proc_every = 1

# IO::checkpoint_dir = "./checkpoints"
# IO::checkpoint_ID = yes
# IO::checkpoint_every = 0
# IO::checkpoint_every_walltime_hours = 3.0
# IO::checkpoint_on_terminate = yes
# CarpetX::checkpoint_method = "openpmd"
# 
# IO::recover_dir = "./checkpoints"
# IO::recover = "autoprobe"
# CarpetX::recover_method = "openpmd"

CarpetX::out_metadata = no

CarpetX::out_silo_vars = "
ADMBaseX::metric
ADMBaseX::curv
ADMBaseX::lapse
ADMBaseX::shift
Z4c::allC
# #TODO Z4cNRPy::allCGF
# CarpetX::regrid_error
"

# CarpetX::out_openpmd_vars = "
# ADMBaseX::metric
# ADMBaseX::curv
# ADMBaseX::lapse
# ADMBaseX::shift
# Weyl::Psi4re
# Weyl::Psi4im
# Z4c::allC
# #TODO Z4cNRPy::allCGF
# CarpetX::regrid_error
# "

CarpetX::out_tsv_vars = "
ADMBaseX::lapse
ADMBaseX::shift
ADMBaseX::metric
ADMBaseX::curv
Z4c::allC
PunctureTracker::pt_loc
PunctureTracker::pt_vel
"

TimerReport::out_every = 512
TimerReport::out_filename = "TimerReport"
TimerReport::output_schedule_timers = no
TimerReport::n_top_timers = 100

\end{verbatim}

\end{document}